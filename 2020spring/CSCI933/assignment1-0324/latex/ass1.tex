\documentclass{article}
\usepackage{fancyhdr}
\usepackage{extramarks}
\usepackage{amsmath}
\usepackage{amsthm}
\usepackage{amsfonts}
\usepackage{tikz}
\usepackage[plain]{algorithm}
\usepackage{algpseudocode}
\usepackage{listings}
\usepackage{amssymb}
\usetikzlibrary{automata,positioning,graphs}

\usepackage{color}

\definecolor{dkgreen}{rgb}{0,0.6,0}
\definecolor{gray}{rgb}{0.5,0.5,0.5}
\definecolor{mauve}{rgb}{0.58,0,0.82}

\lstset{frame=tb,
  language=Python,
  aboveskip=3mm,
  belowskip=3mm,
  showstringspaces=false,
  columns=flexible,
  basicstyle={\small\ttfamily},
  numbers=none,
  numberstyle=\tiny\color{gray},
  keywordstyle=\color{blue},
  commentstyle=\color{dkgreen},
  stringstyle=\color{mauve},
  breaklines=true,
  breakatwhitespace=true,
  tabsize=3
}
%
% Basic Document Settings
%

\topmargin=-0.45in
\evensidemargin=0in
\oddsidemargin=0in
\textwidth=6.5in
\textheight=9.0in
\headsep=0.25in

\linespread{1.1}

\pagestyle{fancy}
\lhead{\hmwkAuthorName}
\chead{\hmwkClass\: \hmwkTitle}
\rhead{\firstxmark}
\lfoot{\lastxmark}
\cfoot{\thepage}

\renewcommand\headrulewidth{0.4pt}
\renewcommand\footrulewidth{0.4pt}

\setlength\parindent{0pt}


% Create Problem Sections
%

\newcommand{\enterProblemHeader}[1]{
    \nobreak\extramarks{}{Problem \arabic{#1} continued on next page\ldots}\nobreak{}
    \nobreak\extramarks{Problem \arabic{#1} (continued)}{Problem \arabic{#1} continued on next page\ldots}\nobreak{}
}

\newcommand{\exitProblemHeader}[1]{
    \nobreak\extramarks{Problem \arabic{#1} (continued)}{Problem \arabic{#1} continued on next page\ldots}\nobreak{}
    \stepcounter{#1}
    \nobreak\extramarks{Problem \arabic{#1}}{}\nobreak{}
}

\setcounter{secnumdepth}{0}
\newcounter{partCounter}
\newcounter{homeworkProblemCounter}
\setcounter{homeworkProblemCounter}{1}
\nobreak\extramarks{Problem \arabic{homeworkProblemCounter}}{}\nobreak{}

%
% Homework Problem Environment
%
% This environment takes an optional argument. When given, it will adjust the
% problem counter. This is useful for when the problems given for your
% assignment aren't sequential. See the last 3 problems of this template for an
% example.
%
\newenvironment{homeworkProblem}[1][-1]{
    \ifnum#1>0
        \setcounter{homeworkProblemCounter}{#1}
    \fi
    \section{Problem \arabic{homeworkProblemCounter}}
    \setcounter{partCounter}{1}
    \enterProblemHeader{homeworkProblemCounter}
}{
    \exitProblemHeader{homeworkProblemCounter}
}

%
% Homework Details
%   - Title
%   - Due date
%   - Class
%   - Section/Time
%   - Instructor
%   - Author
%
\newcommand{\hmwkNum}{1}
\newcommand{\hmwkTitle}{Assignment\ \#\hmwkNum}
\newcommand{\hmwkDueDate}{Dec. 4, 2019}
\newcommand{\hmwkClass}{CSCI933 Machine Learning}
\newcommand{\hmwkClassInstructor}{Chao Sun}
\newcommand{\hmwkAuthorName}{\textbf{Mei Wangzhihui}}
\newcommand{\hmwkAuthorNum}{\textbf{2019124044}}
%
% Title Page
%

\title{
    \vspace{2in}
    \textmd{\textbf{\hmwkClass:\\ \hmwkTitle}}\\
    % \normalsize\vspace{0.1in}\small{Due\ on\ \hmwkDueDate\ at 3:10pm}\\
    % \vspace{0.1in}\large{\textit{\hmwkClassInstructor\ \hmwkClassTime}}
    \vspace{3in}
}

\author{\hmwkAuthorName\ \\ \hmwkAuthorNum}
\date{}

\renewcommand{\part}[1]{\textbf{\large Part \Alph{partCounter}}\stepcounter{partCounter}\\}

%
% Various Helper Commands
%

% Useful for algorithms
\newcommand{\alg}[1]{\textsc{\bfseries \footnotesize #1}}

% For derivatives
\newcommand{\deriv}[1]{\frac{\mathrm{d}}{\mathrm{d}x} (#1)}

% For partial derivatives
\newcommand{\pderiv}[2]{\frac{\partial}{\partial #1} (#2)}

% Integral dx
\newcommand{\dx}{\mathrm{d}x}

% Alias for the Solution section header
\newcommand{\solution}{\textbf{\large Solution}}

% Probability commands: Expectation, Variance, Covariance, Bias
\newcommand{\E}{\mathrm{E}}
\newcommand{\Var}{\mathrm{Var}}
\newcommand{\Cov}{\mathrm{Cov}}
\newcommand{\Bias}{\mathrm{Bias}}

\begin{document}

\maketitle

\clearpage

\begin{homeworkProblem}
$\because S^t=\biggl( \begin{matrix} \frac{1}{\sqrt{2}} & \frac{1}{\sqrt{2}} \\ -\frac{1}{\sqrt{2}} & \frac{1}{\sqrt{2}} \end{matrix} \biggr)=S^{-1} $, $\therefore S$ is an orthogonal matrix. 

As $(SPS^t)^t = SP^tS^t$, $PP^t=SPS^tSP^tS^t=SPP^tS^t=SP(SP)^t$ is a diagonal matrix.

$SPS^t=\biggl( \begin{matrix} \frac{1}{\sqrt{2}} & -\frac{1}{\sqrt{2}} \\ \frac{1}{\sqrt{2}} & \frac{1}{\sqrt{2}} \end{matrix} \biggr) \biggl( \begin{matrix} 1 & 3 \\ 3 & 1 \end{matrix} \biggr) \biggl( \begin{matrix} \frac{1}{\sqrt{2}} & \frac{1}{\sqrt{2}} \\- \frac{1}{\sqrt{2}} & \frac{1}{\sqrt{2}} \end{matrix} \biggr) = $ 

\end{homeworkProblem}

\begin{homeworkProblem}
\noindent
\textbf{(a)}
$SS^t = \biggl[\begin{matrix} cos\alpha & sin\alpha \\ -sin\alpha & cos\alpha\end{matrix} \biggr] \big [\begin{matrix} cos\alpha & -sin\alpha \\ sin\alpha & cos\alpha\end{matrix} \biggr] = I_2$, $S$ is orthogonal.

\textbf{(b)}
% $B^t=SA^tS^t, BB^t=SA(SA)^t, 
$SA=\biggl[\begin{matrix} a_{11}cos\alpha+a_{21}sin\alpha  & a_{12}cos\alpha+a_{22}sin\alpha \\ -a_{11}sin\alpha+a_{21}cos\alpha & -a_{12}sin\alpha+a_{22}cos\alpha \end{matrix} \biggr]$,

$B=SAS^t=\biggl[\begin{matrix} a_{11}cos^2{\alpha}+2a_{12}sin\alpha cos\alpha +a_{22}sin^2\alpha & (a_{22}-a_{11})sin\alpha cos\alpha +a_{12}(cos^2\alpha -sin^2\alpha) \\ (a_{22}-a_{11})sin\alpha cos\alpha + a_{12}(cos^2\alpha -sin^2\alpha) & a_{11}sin^2\alpha-2a_{12}sin\alpha cos\alpha +a_{22}cos^2\alpha \end{matrix} \biggr]$


% $(SA)^t=\biggl[\begin{matrix} a_{11}cos\alpha+a_{21}sin\alpha  & -a_{11}sin\alpha+a_{21}cos\alpha\\ a_{12}cos\alpha+a_{22}sin\alpha & -a_{12}sin\alpha+a_{22}cos\alpha \end{matrix} \biggr]$ 

% As $A$ is symmetric matrix $\biggl[\begin{matrix} a_{11} & a_{12} \\ a_{21} & a_{22} \end{matrix} \biggr], a_{12}=a_{21}$

% $BB^t=\biggl[\begin{matrix} a_{11}^2cos^2\alpha + a_{22}^2sin^2\alpha + a_{12}^2+2a_{12}(a_{11}+a_{22})sin\alpha cos\alpha & (a_{22}^2-a_{11}^2)sin\alpha cos\alpha+(a_{11}a_{21}+a_{22}a_{12})(cos^2\alpha-sin^2\alpha) \\ (a_{22}^2-a_{11}^2)sin\alpha cos\alpha+(a_{11}a_{21}+a_{22}a_{12})(cos^2\alpha-sin^2\alpha) & a_{11}^2cos^2\alpha + a_{22}^2sin^2\alpha + a_{12}^2 - 2a_{12}(a_{11}+a_{22})sin\alpha cos\alpha  \end{matrix} \biggr]$

$tan2\alpha=\frac{2tan\alpha}{1-tan^2\alpha}=\frac{2sin\alpha cos\alpha}{cos^2\alpha-sin^2\alpha}=\frac{2a_{12}}{a_{11}-a_{22}}$,

Set $sin\alpha cos\alpha = pa_{12}, cos^2\alpha-sin^2\alpha = p(a_{11}-a_{22})$

We get $(a_{22}-a_{11})sin\alpha cos\alpha +a_{12}(cos^2\alpha -sin^2\alpha) = pa_{12}(a_{22}-a_{11}) +pa_{12}(a_{11}-a_{22}) = 0 $

Therefore, $B=SAS^t$ is diagonal.

\textbf{(c)}
$Tr[B]=a_{11}cos^2{\alpha}+2a_{12}sin\alpha cos\alpha +a_{22}sin^2\alpha + a_{11}sin^2\alpha-2a_{12}sin\alpha cos\alpha +a_{22}cos^2\alpha = a_{11}+a_{22}=Tr[A]$


\end{homeworkProblem}
\begin{homeworkProblem}
    Set A="coin is fair", B="coin is two-headed", C="heads shows both times"
    If the coin is fair, the probability of two heads is $P(C|A)=\frac{P(C,A)}{P(A)}=\frac14$. If the coin is two-headed, the probability is $P(C|B)=\frac{P(C,B)}{P(B)}=1$. Because $P(A) = P(B)=\frac12$, therefore $P(C,B)=\frac12, P(C,A)=\frac18$, The conditional probability of $P(A|C)=\frac{P(A,C)}{P(C)}=\frac{P(A,C)}{P(B,C)+P(A,C)}=\frac15$ 
\end{homeworkProblem}

\begin{homeworkProblem}
    \textbf{(a)}
    $P(d)=P(d|A)P(A)+P(d|B)P(B)=(\frac{C_{100}^2}{C_{1000}^2} + \frac{C_100^2}{C_{2000}^2})/ \frac{1}{2}\approx 2.5\%$
    
    \textbf{(b)}
    $P(A|d)=\frac{P(A,d)}{P(d)}=\frac{P(A)P(d|A)}{P(d)}=\frac15$

\end{homeworkProblem}

\end{document}