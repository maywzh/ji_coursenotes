%# -*- coding:utf-8 -*-
\documentclass[10pt,aspectratio=169,mathserif]{beamer}		
%设置为 Beamer 文档类型,设置字体为 10pt,长宽比为16:9,数学字体为 serif 风格

%%%%-----导入宏包-----%%%%
\usepackage{ccnu}			%导入 CCNU 模板宏包
\usepackage{xeCJK}			%导入 ctex 宏包,添加中文支持
\usepackage{amsmath,amsfonts,amssymb,bm}   %导入数学公式所需宏包
\usepackage{color}			 %字体颜色支持
\usepackage{graphicx,hyperref,url}
\usepackage{metalogo}	% 非必须
%% 上文引用的包可按实际情况自行增删
%%%%%%%%%%%%%%%%%%	


\beamertemplateballitem		%设置 Beamer 主题

%%%%------------------------%%%%%
\catcode`\。=\active         %或者=13
\newcommand{。}{.}				
%将正文中的“。”号转换为“.”。中文标点国家规范建议科技文献中的句号用圆点替代
%%%%%%%%%%%%%%%%%%%%%

%%%%----首页信息设置----%%%%
\title[Exercise Recommendation Based on GNN]{Research on High School Math Exercise Recommendation Based on Graph Neural Network}
%\subtitle{——这里是副标题}			
%%%%----标题设置


\author[Wangzhihui Mei]{
  Wangzhihui Mei \\\medskip
  {\small \url{maywzh@gmail.com}} \\
  {\small \url{https://maywzh.com/}}}
%%%%----个人信息设置
  
\institute[CCNU-UOW JI]{
  Central China Normal University Wollongong Joint Institude}
%%%%----机构信息

\date[\today]{
  \today}
%%%%----日期信息
  
\begin{document}

\begin{frame}
	\titlepage
\end{frame}				%生成标题页

\section{Outline}
\begin{frame}
	\frametitle{Outline}
	\tableofcontents
\end{frame}				%生成提纲页

\section{Introduction}
\begin{frame}
	\frametitle{Introduction}
	
	\begin{itemize}
		\item {编译方式}
		      \begin{itemize}
			      \item  推荐安装完整版的 TeXLive
			      \item 使用 \XeLaTeX 编译
		      \end{itemize}
		\item 请参考 \LaTeX 和 Beamer 用户文档
		      
		\item 行内数学公式示例 $\sin^2 \theta + \cos^2 \theta = 1$
		\item {行间数学公式示例 \begin{equation}
			      y_{1}=\int \sin x\, {\rm d}x
		      \end{equation}	 }
		\item 基于“华大绿”颜色 \url{http://www.ccnu.edu.cn/}
	\end{itemize}
\end{frame}

\section{内置环境}
\begin{frame}
	\frametitle{内置环境}
	\begin{block}{Slides with \LaTeX}
		Beamer offers a lot of functions to create nice slides using \LaTeX.
	\end{block}
	
	\begin{block}{The basis}
		内部使用以下主题
		\begin{itemize}
			\item split
			\item whale
			\item rounded
			\item orchid
		\end{itemize}
	\end{block}
\end{frame}

\begin{frame}
	\frametitle{带数字列表}
	\begin{enumerate}
		\item This just shows the effect of the style\cite{devlin2019bert}
		\item It is not a Beamer tutorial
		\item Read the Beamer manual for more help
		\item Contact me only concerning the style file
	\end{enumerate}
\end{frame}

\section{结论}
\begin{frame}
	\frametitle{结论}
	
	\begin{itemize}
		\item Easy to use
		\item Good results
	\end{itemize}
\end{frame}

\section{参考文献}
\begin{frame}[allowframebreaks]{References}
	\bibliographystyle{plain}
	%\bibliographystyle{amsalpha}
	%\bibliography{mybeamer} also works
	\bibliography{./ref.bib}
\end{frame}

\end{document}