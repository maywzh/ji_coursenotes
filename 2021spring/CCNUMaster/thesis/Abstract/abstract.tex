% ************************** Thesis Abstract *****************************
% Use `abstract' as an option in the document class to print only the titlepage and the abstract.

%Please list 3-5 keywords, and replace them with "keyword1", "keyword2", "keyword3",...
\begin{abstract}{Graph neural network}{Knowledge point labeling}{Knowledge tracing}{Exercise recommendation}{}
    %高中数学习题推荐系统是一种自适应学习系统,它应用于高中数学学科中,通过跟踪评估学生的知识掌握水平来为学生提供最合适的习题,可以解决人工教育无法解决的个体性知识评估和学习资源推荐。本文立足于这个研究背景,提出一种基于知识追踪的习题推荐系统,它包含三个模块,分别是作为习题库预处理部分的习题知识点标注模块,作为系统核心部分的知识追踪模块,以及最终的习题推荐模块。在习题知识点标注模块中,一种结合图神经网络和基于注意力机制的双向LSTM模块的新型网络架构被提出来用于习题知识点抽取,通过与现有的文本分类模型进行对比,在抽取隐藏知识点和多知识点分类指标上都取得了更强的性能。在知识追踪部分,结合图神经网络的网络传播特性,对DKVMN知识追踪模型进行了改进,并引入学生的个性化答题特征,提出了GKVMN模型。经过与baseline模型对比,提出的模型在多个性能指标上超过了其他模型。在习题推荐模块中,一种基于matching和ranking两阶段的推荐系统模型被提出,它在matching阶段结合多种matching策略来生成候选推荐习题集,在ranking阶段结合知识追踪模块生成的学生知识状态熟练度来为生成的习题排序。通过与传统的推荐算法进行对比,并设计对比指标,结果显示了提出的推荐系统模型具备自适应学习的能力。
    High school mathematics exercise recommendation system is an adaptive learning system which is applied to high school mathematics subject to provide students with the most suitable exercises by tracing and assessing their knowledge mastery level, which can solve the problem of individual knowledge assessment and learning resource recommendation that cannot be solved by manual education. Based on this research background, a knowledge-tracing-based exercise recommendation system is proposed, consisting of three modules: the exercise knowledge point annotation module as the pre-processing part of the exercise database, the knowledge tracing the core part of the system, and the final exercise recommendation module. In the exercise knowledge point annotation module, a novel network architecture combining graph neural network and attention mechanism-based bidirectional LSTM module is proposed for exercise knowledge point extraction, which achieves more robust performance in extracting hidden knowledge points and multi-knowledge point classification tasks by comparing with existing text classification models. In the knowledge tracing part, the DKVMN knowledge tracing model is improved by combining the network propagation characteristics of graph neural networks, and the GKVMN model is proposed by introducing the personalized answer characteristics of students. After comparing with the baseline model, the proposed model outperforms other models in several performance metrics. In the exercise recommendation module, a two-stage recommendation system model based on matching and ranking is proposed, which combines multiple matching strategies to generate candidate recommended sets of exercises in the matching stage and ranks the generated exercises in the ranking stage by combining the student knowledge state proficiency generated by the knowledge tracing module. Comparing with traditional recommendation algorithms and designing comparison metrics shows that the proposed recommendation system model meets the intended design requirements and performance specifications.
\end{abstract}
