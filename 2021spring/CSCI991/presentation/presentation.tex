\documentclass{beamer}

\mode<presentation> {

\usetheme{default}
%\usetheme{AnnArbor}
%\usetheme{Antibes}
%\usetheme{Bergen}
%\usetheme{Berkeley}
%\usetheme{Berlin}
%\usetheme{Boadilla}
%\usetheme{CambridgeUS}
%\usetheme{Copenhagen}
%\usetheme{Darmstadt}
%\usetheme{Dresden}
%\usetheme{Frankfurt}
%\usetheme{Goettingen}
%\usetheme{Hannover}
%\usetheme{Ilmenau}
%\usetheme{JuanLesPins}
%\usetheme{Luebeck}
%\usetheme{Madrid}
%\usetheme{Malmoe}
%\usetheme{Marburg}
%\usetheme{Montpellier}
%\usetheme{PaloAlto}
%\usetheme{Pittsburgh}
%\usetheme{Rochester}
%\usetheme{Singapore}
%\usetheme{Szeged}
%\usetheme{Warsaw}

%\usecolortheme{albatross}
%\usecolortheme{beaver}
%\usecolortheme{beetle}
%\usecolortheme{crane}
%\usecolortheme{dolphin}
%\usecolortheme{dove}
%\usecolortheme{fly}
%\usecolortheme{lily}
%\usecolortheme{orchid}
%\usecolortheme{rose}
%\usecolortheme{seagull}
%\usecolortheme{seahorse}
%\usecolortheme{whale}
%\usecolortheme{wolverine}
%\usecolortheme{structure}

%\setbeamertemplate{footline} % To remove the footer line in all slides uncomment this line
%\setbeamertemplate{footline}[page number] % To replace the footer line in all slides with a simple slide count uncomment this line

%\setbeamertemplate{navigation symbols}{} % To remove the navigation symbols from the bottom of all slides, uncomment this line
}
\usepackage{fontspec}
\usepackage{bibentry}
\usepackage{graphicx} % Allows including images
\usepackage{booktabs} % Allows the use of \toprule, \midrule and \bottomrule in tables
\usepackage{url}

\setmainfont{ArialMT}
%\setbeamertemplate{frametitle}[default][center]

% 加导航条
%\useoutertheme[width=3\baselineskip,right]{sidebar}
% 目录标数字
\setbeamertemplate{section in toc}[sections numbered] 
% 无序列表用实心点
\setbeamertemplate{itemize item}{\(\bullet \)}
% 设置每页标题格式
% \setbeamertemplate{frametitle}
%   {\vspace{-0.5cm}
%    \insertframetitle
%    \vspace{-0.5cm}}
% 去掉下面没用的导航条
%\setbeamertemplate{navigation symbols}{}
% 设置页脚格式
% \makeatother
% \setbeamertemplate{footline}
% {
%   \leavevmode%
%   \hbox{%
%   \begin{beamercolorbox}[wd=.4\paperwidth,ht=2.25ex,dp=1ex,center]{author in head/foot}%
%     \usebeamerfont{author in head/foot}\insertshortauthor
%   \end{beamercolorbox}

%   \begin{beamercolorbox}[wd=.6\paperwidth,ht=2.25ex,dp=1ex,center]{title in head/foot}%
%     \usebeamerfont{title in head/foot}\insertshorttitle\hspace*{13em}
%     \insertframenumber{} / \inserttotalframenumber\hspace*{0ex}
%   \end{beamercolorbox}}

%   \vskip0pt%
% }
% \makeatletter

% 定义颜色
\definecolor{alizarin}{rgb}{0.82, 0.1, 0.26} % 红色
%\definecolor{DarkFern}{HTML}{407428} % 绿色
%\colorlet{main}{DarkFern!100!white} % 第一种设置方法
\colorlet{main}{red!70!black} % 第二种设置方法
\definecolor{bistre}{rgb}{0.24, 0.17, 0.12} % 黑色
\definecolor{mygrey}{rgb}{0.52, 0.52, 0.51} % 灰色
%\colorlet{main}{green!50!black}
\colorlet{text}{bistre!100!white}

% 不同元素指定不同颜色,fg是本身颜色,bg是背景颜色,!num!改变数值提供渐变色
\setbeamercolor{title}{fg=main}
\setbeamercolor{frametitle}{fg=main}
\setbeamercolor{section in toc}{fg=text}
\setbeamercolor{normal text}{fg=text}
\setbeamercolor{block title}{fg=main,bg=mygrey!14!white}
\setbeamercolor{block body}{fg=black,bg=mygrey!10!white}
%\setbeamercolor{block body}{fg=text}
\setbeamercolor{qed symbol}{fg=main} % 证明结束后的框颜色
\setbeamercolor{math text}{fg=black}
% 设置页脚对应位置颜色
% \setbeamercolor{author in head/foot}{fg=black, bg=mygrey!5!white}
% \setbeamercolor{title in head/foot}{fg=black, bg=mygrey!5!white}
\setbeamercolor{structure}{fg=main, bg=mygrey!10!white} % 设置sidebar颜色

% 左右页间距的排版
% \def\swidth{2.3cm}
% \setbeamersize{sidebar width right=\swidth}
% \setbeamersize{sidebar width left=\swidth}
% \setbeamerfont{title in sidebar}{size=\scriptsize}
% \setbeamerfont{section in sidebar}{size=\tiny}

% \setbeamertemplate{frametitle}
% {\begin{beamercolorbox}[wd=\paperwidth]{frametitle}
%       \strut\hspace{0.5em}\insertframetitle\strut
%       \hfill
%       \raisebox{-2mm}{\includegraphics[width=1cm]{figures/JI-logo.png}}
%     \end{beamercolorbox}
% }
\setbeamertemplate{bibliography item}[text] %参考文献图标改普通
%----------------------------------------------------------------------------------------
%	TITLE PAGE
%----------------------------------------------------------------------------------------

\title[Exercise Recommendation]{Research on High School Math Exercise Recommendation Based on Graph Neural Network} % The short title appears at the bottom of every slide; the full title is only on the title page

\author{Wangzhihui Mei} % Your name
\institute[UOW] 
{
University of Wollongong \\ % Your institution for the title page
\medskip
\textit{maywzh@gmail.com} % Your email address
}
\date{\today} % Date, can be changed to a custom date

\begin{document}

\begin{frame}
  \titlepage % Print the title page as the first slide
\end{frame}

\begin{frame}
  \frametitle{Overview} % Table of contents slide, comment this block out to remove it
  \tableofcontents % Throughout your presentation, if you choose to use \section{} and \subsection{} commands, these will automatically be printed on this slide as an overview of your presentation
\end{frame}

%----------------------------------------------------------------------------------------
%	PRESENTATION SLIDES
%----------------------------------------------------------------------------------------

%------------------------------------------------
\section{Introduction}
%------------------------------------------------
\subsection{Research Background}
\begin{frame}
  \frametitle{Research Background}
  \begin{itemize}
    \item Knowledge State Monitoring
    \item Learning Resource Recommendation
    \item High School Math (Chinese)
  \end{itemize}
\end{frame}

%------------------------------------------------
\subsection{Existing Problems}
\begin{frame}
  \frametitle{Existing Problems}
  \begin{description}
    \item[Inappropriate Recommendation] Exercise recommendation is not based on knowledge mastery
    \item[Disorganized exercise] Labelling knowledge for exercises lacking knowledge tags
    \item[Knowledge evaluation] The difficulty for obtaining knowledge mastery proficiency of the student
    \item[Exercise recommendation] How to recommend appropriate exercises according to their knowledge status
  \end{description}
\end{frame}
%------------------------------------------------
\subsection{Research Cores}
\begin{frame}
  \frametitle{Research Cores}
  \begin{block}{Exercise knowledge labeling}
    A multi-knowledge point labeling algorithm for high school mathematics exercises based on bidirectional LSTM (Bi-LSTM)~\cite{chen2017improving} and graph convolutional neural network (GCN)~\cite{kipf2016semi}.
  \end{block}
  \begin{block}{Knowledge tracing}
    A knowledge tracing model based on Transformer~\cite{vaswani2017attention} architecture with graph attention network embedding.
  \end{block}
  \begin{block}{Exercise recommendation}
    A mathematical exercise recommendation model based on Matching-Ranking~\cite{segev2009context} algorithm.
  \end{block}
\end{frame}
%------------------------------------------------
\section{Proposed Model}
%------------------------------------------------
\subsection{Exercise Knowledge Labelling}
\begin{frame}
  \frametitle{Exercise Knowledge Labelling}
  \framesubtitle{Architecture}
  \begin{figure}
    \includegraphics[height=0.8\textheight]{figures/ch2-model-architecture.pdf}
    \caption{Model architecture}\label{fig:ch2-archi}
  \end{figure}
\end{frame}

% \begin{frame}
%   \frametitle{Exercise Knowledge Labelling}
%   \framesubtitle{Details}
%   \begin{columns}[c] % The "c" option specifies centered vertical alignment while the "t" option is used for top vertical alignment
%     \column{.45\textwidth} % Left column and width
%     \textbf{Modules}
%     \begin{enumerate}
%       \item \textbf{BERT~\cite{devlin2019bert} Embedding Layer}
%       \item Attentional Bi-LSTM Text Representation
%       \item GCN-based Classifiers
%     \end{enumerate}
%     \column{.5\textwidth} % Right column and width
%     \begin{figure}
%       \includegraphics[width=1.0\textwidth]{figures/ch2-bert-model.pdf}
%     \end{figure}
%   \end{columns}
%   Thanks for opensource pre-trained Chinese language models from Cui et.al.~\cite{wwmbertgithub,chinese-bert-wwm}.
% \end{frame}

% \begin{frame}
%   \frametitle{Exercise Knowledge Labelling}
%   \framesubtitle{Details}
%   \begin{columns}[c] % The "c" option specifies centered vertical alignment while the "t" option is used for top vertical alignment
%     \column{.45\textwidth} % Left column and width
%     \textbf{Modules}
%     \begin{enumerate}
%       \item BERT~\cite{devlin2019bert} Embedding Layer
%       \item \textbf{Attentional Bi-LSTM Text Representation}
%       \item GCN-based Classifiers
%     \end{enumerate}
%     \column{.5\textwidth} % Right column and width
%     \begin{figure}
%       \includegraphics[width=1.0\textwidth]{figures/ch2-model-bilstm.pdf}
%     \end{figure}
%   \end{columns}
% \end{frame}

% \begin{frame}
%   \frametitle{Exercise Knowledge Labelling}
%   \framesubtitle{Details}
%   \begin{columns}[c] % The "c" option specifies centered vertical alignment while the "t" option is used for top vertical alignment
%     \column{.45\textwidth} % Left column and width
%     \textbf{Modules}
%     \begin{enumerate}
%       \item BERT~\cite{devlin2019bert} Embedding Layer
%       \item Attentional Bi-LSTM Text Representation
%       \item \textbf{GCN-based Classifiers}
%     \end{enumerate}
%     \column{.5\textwidth} % Right column and width
%     \begin{figure}
%       \includegraphics[width=1.0\textwidth]{figures/ch2-gcn-ov.pdf}
%     \end{figure}
%   \end{columns}
% \end{frame}

%------------------------------------------------
\subsection{Knowledge Tracing}

\begin{frame}
  \frametitle{Knowledge Tracing}
  \framesubtitle{Problem}
  \begin{figure}
    \includegraphics[width=1.0\textwidth]{figures/ch3-model-ktdes.pdf}
  \end{figure}
\end{frame}

\begin{frame}
  \frametitle{Knowledge Tracing}
  \framesubtitle{Architecture}
  \begin{figure}
    \includegraphics[height=0.8\textheight]{figures/ch3-overview.pdf}
  \end{figure}
\end{frame}

\begin{frame}
  \frametitle{Knowledge Tracing}
  \framesubtitle{Detail}
  \begin{figure}
    \includegraphics[width=0.9\textwidth]{figures/ch3-gat-kq.pdf}
  \end{figure}
\end{frame}

% \begin{frame}
%   \frametitle{Knowledge Tracing}
%   \framesubtitle{Detail}

% \end{frame}


\subsection{Exercise Recommendation}
\begin{frame}
  \frametitle{Exercise Recommendation}
  \framesubtitle{Architecture}
  \begin{figure}
    \includegraphics[width=0.9\textwidth]{figures/ch4-ov.pdf}
    \caption{Recommendation Architecture}
  \end{figure}
\end{frame}

\begin{frame}
  \frametitle{Exercise Recommendation}
  \framesubtitle{Detail}
  \begin{figure}
    \includegraphics[height=0.7\textheight]{figures/ch4-matching.pdf}
    \caption{Matching Phase}
  \end{figure}
\end{frame}

\begin{frame}
  \frametitle{Exercise Recommendation}
  \framesubtitle{Detail}
  \begin{figure}
    \includegraphics[width=0.9\textwidth]{figures/ch4-ranking.pdf}
    \caption{Ranking Phase}
  \end{figure}
\end{frame}

%------------------------------------------------
\section{Experiment Design}
\subsection{Exercise Knowledge Labelling}
\begin{frame}
  \frametitle{Experiment Design}
  \framesubtitle{Exercise Knowledge Labelling}
  \begin{itemize}
    \item Compared with several baseline models
    \item The number of occurrences of the knowledge point label \(\tau^{(KP)} \)
  \end{itemize}
  \begin{table}[htbp!]
    \centering
    \caption{Setting of Experiment}\label{tbl:ch2-ex1}
    \begin{tabular}{cccc}%{cp{2cm}<{\centering}p{2cm}<{\centering}p{2cm}<{\centering}}
      \toprule
      \text{\(\tau^{(KP)} \)} & \(|\mathbf{L}_{\tau^{(KP)}}|\) & \(|\mathbf{E}_{\tau^{(KP)}}| \) & \(\overline{L}\) \\
      \midrule
      200                     & 2                              & 463                             & 1.21             \\
      100                     & 22                             & 1376                            & 1.55             \\
      50                      & 29                             & 2237                            & 1.42             \\
      10                      & 57                             & 3158                            & 1.35             \\
      \bottomrule
    \end{tabular}
  \end{table}
\end{frame}

\begin{frame}
  \frametitle{Experiment Design}
  \framesubtitle{Knowledge Tracing}
  \begin{block}{Basic Method}
    Compare with other KT baseline models BKT~\cite{yudelson2013individualized}、DKT~\cite{piech2015deep}、DKVMN~\cite{chen2017improving} and GKT~\cite{nakagawa2019graph}
  \end{block}
  \begin{table}[htbp!]
    \centering
    \caption{Dataset Statistics}\label{tbl:ch2-tb1}
    \scalebox{0.7}{
      \begin{tabular}{ccccc}
        \toprule
        Dataset   & \#students & \#exercises & \#knowledge points & \#interactions \\
        \midrule
        ASSIST15  & 19,917     & 102,396     & 100                & 709K           \\
        ASSIST17  & 1,709      & 4,117       & 102                & 943K           \\
        STATICS11 & 333        & 1,223       & 156                & 189K           \\
        \bottomrule
      \end{tabular}}
  \end{table}
\end{frame}

\begin{frame}
  \frametitle{Experiment Design}
  \framesubtitle{Exercise Recommendation}
  \begin{itemize}
    \item Compared with conventional Collaborative Filtering and Random Recommendation
    \item Using adapted KT dataset for testing
    \item Check if the selected exercise is in the final recommendation list
  \end{itemize}
\end{frame}


\section{Result and Analysis}
\begin{frame}
  \frametitle{Result}
  \framesubtitle{Exercise Knowledge Labeling}
  \begin{table}[htbp!]
    \caption{Result comparison (\(\tau^{(KP)}=200 \))}\label{tbl:bsline1}
    \centering
    \scalebox{0.7}{
      \begin{tabular}{cccccccc}
        \toprule
        Metrics      & \(\operatorname{F1}_{macro}\) & \(\operatorname{F1}_{micro}\) & \(\operatorname{Acc}_{ML}\) & \(\operatorname{HmLoss}\) & \(\operatorname{F1}_{ML}\) \\
        \midrule
        NB           & 75.3                          & 74.2                          & 69.6                        & 18.2                      & 73.6                       \\
        ML-KNN       & 77.1                          & 76.2                          & 73.2                        & 17.4                      & 76.3                       \\
        CNN+word2vec & 79.5                          & 78.4                          & 76.6                        & 14.2                      & 79.6                       \\
        CNN+BERT     & 80.1                          & \textbf{79.9}                 & 76.9                        & 13.7                      & 79.5                       \\
        Proposed     & \textbf{80.9}                 & 79.1                          & \textbf{77.3}               & \textbf{13.1}             & \textbf{80.7}              \\
        \bottomrule
      \end{tabular}}
  \end{table}
  \begin{table}
    \centering
    \caption{Result comparison (\(\tau^{(KP)}=100 \))}\label{tbl:bsline2}
    \scalebox{0.7}{
      \begin{tabular}{cccccccc}
        \toprule
        Metrics      & \(\operatorname{F1}_{macro}\) & \(\operatorname{F1}_{micro}\) & \(\operatorname{Acc}_{ML}\) & \(\operatorname{HmLoss}\) & \(\operatorname{F1}_{ML}\) \\
        \midrule
        NB           & 71.2                          & 72.1                          & 67.2                        & 16.2                      & 71.8                       \\
        ML-KNN       & 73.2                          & 72.3                          & 69.1                        & 15.9                      & 74.7                       \\
        CNN+word2vec & 74.3                          & 74.4                          & 72.3                        & 13.2                      & 75.2                       \\
        CNN+BERT     & 74.4                          & 74.6                          & 72.3                        & 13.1                      & 75.1                       \\
        Proposed     & \textbf{75.5}                 & \textbf{75.7}                 & \textbf{73.1}               & \textbf{12.7}             & \textbf{74.9}              \\
        \bottomrule
      \end{tabular}}
  \end{table}
\end{frame}

\begin{frame}
  \frametitle{Result}
  \framesubtitle{Exercise Knowledge Labeling}
  \begin{table}
    \centering
    \caption{Result comparison (\(\tau^{(KP)}=50 \))}\label{tbl:bsline3}
    \scalebox{0.7}{
      \begin{tabular}{cccccccc}
        \toprule
        Metrics      & \(\operatorname{F1}_{macro}\) & \(\operatorname{F1}_{micro}\) & \(\operatorname{Acc}_{ML}\) & \(\operatorname{HmLoss}\) & \(\operatorname{F1}_{ML}\) \\
        \midrule
        NB           & 52.3                          & 53.0                          & 42.1                        & 9.2                       & 51.9                       \\
        ML-KNN       & 44.2                          & 43.9                          & 23.5                        & 10.1                      & 42.1                       \\
        CNN+word2vec & 56.1                          & \textbf{57.3}                 & 46.2                        & 8.2                       & 56.5                       \\
        CNN+BERT     & 56.2                          & 56.8                          & \textbf{47.0}               & \textbf{8.1}              & 56.1                       \\
        Proposed     & \textbf{57.1}                 & 57.2                          & 45.2                        & 8.6                       & \textbf{57.5}              \\
        \bottomrule
      \end{tabular}}
  \end{table}

  \begin{table}
    \centering
    \caption{Result comparison (\(\tau^{(KP)}=10 \))}\label{tbl:bsline4}
    \scalebox{0.7}{
      \begin{tabular}{cccccccc}
        \toprule
        Metrics      & \(\operatorname{F1}_{macro}\) & \(\operatorname{F1}_{micro}\) & \(\operatorname{Acc}_{ML}\) & \(\operatorname{HmLoss}\) & \(\operatorname{F1}_{ML}\) \\
        \midrule
        NB           & 36.5                          & 37.1                          & 26.1                        & 4.2                       & 36.5                       \\
        ML-KNN       & 30.1                          & 32.1                          & 29.1                        & 3.6                       & 32.5                       \\
        CNN+word2vec & 36.7                          & 38.2                          & 37.5                        & 3.5                       & 37.2                       \\
        CNN+BERT     & 36.9                          & \textbf{38.6}                 & \textbf{38.6}               & \textbf{3.5}              & 37.5                       \\
        Proposed     & \textbf{37.1}                 & 38.3                          & 35.4                        & 3.8                       & \textbf{38.6}              \\
        \bottomrule
      \end{tabular}}
  \end{table}
\end{frame}

% \begin{frame}
%   \frametitle{Result}
%   \framesubtitle{Exercise Recommendation}
%   \begin{figure}
%     \includegraphics[width=0.9\textwidth]{figures/ch2_fg1.png}
%   \end{figure}
% \end{frame}

\begin{frame}
  \frametitle{Result}
  \framesubtitle{Knowledge Tracing}
  \begin{table}[htbp!]
    \centering
    \caption{AUC results (\%) over three datasets}\label{tbl:ch3-tb2}
    \begin{tabular}{cccc}
      \toprule
      Model    & ASSIST15                    & ASSIST17                   & STATICS11                  \\
      \midrule
      BKT      & \(62.01\pm 0.03 \)          & \(65.30\pm 0.01\)          & \(64.21\pm 0.01\)          \\
      DKT      & \(70.83\pm 0.03 \)          & \(72.66\pm 0.01\)          & \(72.46\pm 0.06\)          \\
      DKVMN    & \(71.06\pm 0.03 \)          & \(72.78\pm 0.02\)          & \(72.67\pm 0.03\)          \\
      GKT      & \(72.12\pm 0.02 \)          & \(72.85\pm 0.01\)          & \(72.57\pm 0.01\)          \\
      \midrule
      Proposed & \(\mathbf{72.62\pm 0.02} \) & \(\mathbf{72.89\pm 0.02}\) & \(\mathbf{72.73\pm 0.02}\) \\
      \bottomrule
    \end{tabular}
  \end{table}
\end{frame}

\begin{frame}
  \frametitle{Result and Analysis}
  \framesubtitle{Exercise Recommendation}
  a given judgment threshold \(\tau \) and the output probability \(p\) of KT and the judgment of error-prone of one exercise \(O\) satisfy:
  \begin{align}
    O =\{\begin{array}{ll}
      1, & \text{ if } \tau<p      \\
      0, & \text{ if } \tau \geq p
    \end{array}\notag
  \end{align}
  \begin{table}[htbp!]
    \caption{Recommendation Experiment Result}\label{table:ch4-exp-result}
    \centering
    \begin{tabular}{c c c}
      \toprule
      Model                    & ACC    & AUC    \\
      \midrule
      CF                       & 0.5375 & 0.5239 \\
      Random                   & 0.5073 & 0.5012 \\
      \midrule
      Proposed \((\tau=0.45)\) & 0.6542 & 0.6878 \\
      Proposed \((\tau=0.50)\) & 0.6839 & 0.7375 \\
      Proposed \((\tau=0.55)\) & 0.6657 & 0.7039 \\
      \bottomrule
    \end{tabular}
  \end{table}
\end{frame}


\section{Conclusion}
\begin{frame}
  \frametitle{Conclusion}
  \begin{itemize}
    \item The three modules of the proposed model satisfy the requirements of the design
    \item The proposed model achieves better performance compared with baseline models.
  \end{itemize}
\end{frame}
%------------------------------------------------

\begin{frame}[allowframebreaks]{References}
  \bibliographystyle{plain}
  %\bibliographystyle{amsalpha}
  %\bibliography{mybeamer} also works
  \bibliography{./ref.bib}
\end{frame}

%------------------------------------------------

\begin{frame}
  \Huge{\centerline{The End}}
\end{frame}

%----------------------------------------------------------------------------------------

\end{document}