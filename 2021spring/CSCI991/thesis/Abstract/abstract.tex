% ************************** Thesis Abstract *****************************
% Use `abstract' as an option in the document class to print only the titlepage and the abstract.

%Please list 3-5 keywords, and replace them with "keyword1", "keyword2", "keyword3",...
\begin{abstract}{Graph Neural Network}{Knowledge Point Labeling}{Knowledge Tracking}{Recommended Exercises}{}
    After 2010, artificial intelligence technology has gradually become a research hotspot in the field of computer technology. In particular, the advent of AlphaGo has aroused great concern in the industry for the prospect of artificial intelligence. This has brought about the explosive development of the industry, and also raised a large number of research topics. In artificial intelligence-related research, various algorithm innovations, theoretical breakthroughs and model applications are emerging one after another, laying the foundation for the intelligence of various industries. The application of artificial intelligence technology in the field of education has also given birth to the emergence of the concept of intelligent education. Among them, adaptive learning is one of the popular application fields in intelligent education\cite{ma2017adalearn}. Adaptive learning models generally track the learning status of students by combining big data analysis of massive student group learning data and precise data analysis of target student individual data, targeting on personalizing the learning path according to the individual characteristics of the students and the proficiency of knowledge mastery\cite{soltani2019adaptive}. Adaptive learning technology can use automated machine learning algorithms to complete student evaluation and teaching plans that required a lot of manual labor in the past, which can systematically alleviate the current scarcity and uneven distribution of domestic educational resources as well as reduce the burden on education practitioners and students. It also has great development prospects and commercial value. There are more and more artificial intelligence research teams and intelligent education technology companies on the market focusing on the development and application of adaptive learning tools. Some smart educational technology companies have used adaptive learning as the core function or main selling point of their products. Adaptive educational technology can comprehensively analyze students' individual level learning ability, knowledge proficiency and group level popular learning resources, error-prone questions, etc., so that the most suitable learning path and learning resources such as exercises, materials, and knowledge points can be pushed to student. The system automatically adjusts the knowledge focus of the pushed learning resources according to the students' knowledge status to prevent repetitive practice of already mastered knowledge points or lack of practice of unmastered knowledge points. On the one hand, teachers can analyze the knowledge mastery proficiency of the whole class based on the data or visual charts output by the system to create a learning status assessment report for each student and adjust the overall teaching plan in an adaptive manner. On the other hand, students can use the system to analyze their knowledge weaknesses and thus targeted appropriate exercises. Thus, adaptive learning is one of the potentially feasible solutions to the problem of ``automatic assessment of students'' knowledge mastery status and instructional program generation'' in online education.

    The purpose of this paper is to propose a knowledge-tracking-based model for recommending high school mathematics exercises, using the subject of high school mathematics as the main research context. In the subject of high school mathematics, practice exercises are the main means for students to improve their learning ability. However, in the current high school mathematics teaching, teachers or students need to find suitable exercises to practice from a huge library of exercises, which are often too large, highly repetitive and confusingly organized. There are quite a lot of low-quality unlabeled exercises in the exercise bank, which need to be manually labeled with knowledge points. Some students study through excessive exercises tactic, but this is less efficient and often results in repetition of familiar knowledge and avoidance of unfamiliar knowledge. In order to improve the effectiveness of the exercises performed by students and thus enhance the proficiency and comprehensiveness of knowledge acquisition, experienced teaching staff is needed to conduct analysis of students' knowledge status and to select appropriate exercises from the exercise bank for recommendation. The method is uneconomical and inefficient because of its high manual workload, its reliance on expert a priori knowledge, and its inclusion of a large amount of repetitive work. In addition, the traditional exercise recommendation takes the student group as the minimum granularity, but does not recommend for the knowledge mastery proficiency of specific students, which ignores the problem that different students have different learning abilities, so the recommendation is less fine-grained and ineffective for most students. The method is uneconomical and inefficient because of its high manual workload, its reliance on expert a priori knowledge, and its inclusion of a large amount of repetitive work.In addition, the traditional exercise recommendation takes the student group as the minimum granularity, but does not recommend for the knowledge mastery proficiency of specific students, which ignores the problem that different students have different learning abilities, so the recommendation is less fine-grained and ineffective for most students. In order to improve the problems of traditional exercise recommendation methods, knowledge tracking techniques can be applied to track students' learning and thus target automated exercise recommendations. The goal of this paper is to design an exercise recommendation system based on knowledge point annotation, knowledge tracking and resource recommendation techniques, and thus introduce an intelligent adaptive learning solution in terms of exercise recommendation.


    This method requires much manual work, relies on experts' prior knowledge, and does not make recommendations for students' individualized knowledge mastery. The recommendation effect is plough to improve this kind of situation. The application of knowledge tracking technology can recommend exercises based on the students' knowledge mastery. This article aims to design an exercise recommendation system that combines knowledge point labelling, knowledge tracking, and resource recommendation technology to form an intelligent adaptive learning solution in exercise recommendation.

    The high school math learning question recommendation system proposed in this paper includes three modules: the exercise knowledge point labelling module, the knowledge tracking module, and the recommendation module. The exercise knowledge point mining module's function is to label the knowledge points for the exercises that have not been marked so as to replace the traditional manual labelling with machine automation. Knowledge point labelling is the pre-work of exercise recommendation, and the exercises that have been labelled with knowledge can be used as the exercises of the knowledge tracking system to be embedded in the learning data source. The knowledge tracking module obtains and calculates the student's knowledge state's representation by tracking the student's question record. The knowledge state vector represents the student's knowledge and skill mastery and is the core part of the entire system. In the final exercise recommendation module, input the knowledge vector and exercise embedding representation obtained in the previous period, and combine collaborative filtering and related recommendation models to perform preliminary filtering and fine recommendation of exercises, thereby realizing exercise recommendation based on knowledge status.
    \begin{itemize}
        \item Chapter 2 proposes a multi-knowledge point labelling method for high school math learning questions based on two-way LSTM and graph neural network. The exercise knowledge point labelling module includes two sub-modules: exercise text mining and knowledge point labelling. This paper mainly uses exercise text information mining to extract knowledge points. Therefore, a two-way LSTM network based on the attention mechanism is used for exercise text mining. The exercises first undergo preprocessing steps such as word segmentation, cleaning, and regularization to obtain the word sequence. At the same time, filter out the interference of a lot of irrelevant information. Then there is a calculation to reduce the impact of matrix sparseness. Using word embedding instead of simple one-hot encoding as a word embedding can prevent problems such as the disaster of dimensionality caused by the sparse input word vector matrix. Besides, the embedded learning method can also characterize the hidden dependence relationship between word vectors, which is beneficial to construct the dependence relationship between knowledge points. Next, the two-way LSTM model is used to extract text information, which can better use context information for text classification. To capture and solve the dependency between knowledge points, this paper proposes a multi-label knowledge point labelling model based on a graph convolutional network (GCN). Each label is represented by the embedding of knowledge points. After multiple rounds of iterative learning, the label map Mapped to a set of internally dependent knowledge point classifiers. These classifiers are then applied to the text description of the exercises extracted by the previous sub-network so as to realize the task of multi-knowledge point labelling. In the experimental stage, through experiments on the actual high school math learning question data set, the method proposed in this paper is compared with a series of benchmark models and SOTA models, and a series of multi-label classification index parameters are used to compare and evaluate model performance. The experimental results show that this method has achieved superior performance on the problem sets with more complex knowledge point relationships.

        \item Chapter 3 proposes a knowledge tracking model based on graph attention network and Transformer architecture. The model uses graph attention network to learn the relationship between knowledge points between exercises. The traditional knowledge tracking model optimizes the lack of representation of the complex relationship between knowledge points between exercises. It solves the following problems: (1) The traditional model cannot output the student's knowledge state, but can only output the correct probability of the next exercise, which makes it difficult to combine the recommendation model to recommend exercises. (2) The traditional model models knowledge points as mutually independent relationships or hierarchical relationships, while ignoring the complex graph-like relationships between knowledge points, and thus performs poorly on data with complex dependencies on knowledge points. The proposed model combines the powerful representation and learning capabilities of the graph neural network for data in non-Euclidean spaces, and the Transformer model's information encoding and decoding capabilities for serialized exercise data, which can be used for knowledge on exercise sets with complex knowledge dependence. The tracking task performed well. In the experimental stage, the performance of the model proposed in this paper is compared with the benchmark model and the recently proposed SOTA model through experiments on the actual knowledge tracking public data set. The experimental results show that on the public data set, this model has achieved better or similar results in terms of evaluation parameters compared to most models.
        \item Chapter 4 proposes a two-stage exercise recommendation system based on collaborative filtering and neural networks. In the first stage, a part of learner-related exercises similar to the target student is filtered through the collaborative filtering algorithm as a preliminary filtering exercise set, and the second stage is through the neural network. The network model sorts the knowledge status, student learning ability and exercises as the final recommended exercise list. This chapter filters out part of the exercises that are most suitable for recommendation through the first stage of collaborative filtering, and solves the problem of recommendation efficiency in large-scale exercise sets. In the second stage of the ranking process, the knowledge state vector obtained from the knowledge tracking module and the label vector of the exercises are input into the final ranking model, and a priority sequence of exercise recommendation is input. The experimental results show that the proposed model has achieved excellent results in terms of efficiency and recommendation effect.
    \end{itemize}

    This paper analyzes the requirements of the system, rationalizes the entire recommendation system into multiple modules, and designs different neural network models and algorithms for each module to achieve and optimize the above three modules. It has both algorithm design and experimental verification methods. A certain degree of innovation. After experimental verification, it has better performance than similar models.
\end{abstract}
