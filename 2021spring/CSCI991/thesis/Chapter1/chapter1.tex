%*******************************************************************************
%*********************************** First Chapter *****************************
%*******************************************************************************

\chapter{Introduction}  %Title of the First Chapter

\ifpdf
	\graphicspath{{Chapter1/Figs/Raster/}{Chapter1/Figs/PDF/}{Chapter1/Figs/}}
\else
	\graphicspath{{Chapter1/Figs/Vector/}{Chapter1/Figs/}}
\fi


%********************************** %First Section  **************************************
\section{Research Background and Significance} %Section - 1.1
%
%人工智能的技术研究和商业化应用部署在近年来产生了加速发展的趋势。各种基于人工智能和大数据分析相关算法的部署上线在加速企业、组织数字化,改善产业链结构和提高信息利用效率方面发挥了积极作用。教育行业作为传统的服务行业,也存在相当大的智能化改进的空间。在传统教育模式中,学生往往以班级为基础教育集体,因而在一个班级中,所有的教学活动的粒度也以班级作为基本单位,导致学生所学习的内容与其需求不完全匹配,并且已经掌握的知识点过度练习和未掌握的知识点缺乏练习的情况时有发生。这造成了学生出现厌学、学习焦虑等状况,对学习效率和学习效果造成了较大影响。另一方面,对于教师而言,普遍性的个体知识状态精细监测和评估需要巨大的工作量,因此教师往往只关注一部分学生,导致多数学生的学习情况被忽略。这对于学生的学习积极性和学习状况具有较大的影响。

%而在近年来,我国发布了一系列促进人工智能在教育领域的应用的政策。通过人工智能技术来进行学生学习情况监测和追踪,可以大大改善教育质量和效率,从而促成智慧教育的实现。在智慧教育中,自适应学习是一个已经经过大量实践和验证过的商业模式,相当多的在线教育平台部署了自适应学习系统服务。它运用数据分析、深度学习等技术,分析学生的学习行为数据,自动学生进行知识状态的评估和追踪,并结合学生的潜能、优势等等个性化信息向学生提供学习路径规划和学习资源推荐服务。它能够帮助学生提升学习的有效性和效率,在降低教师教学压力的同时提升教师的教学质量。其中,习题推荐系统是一个自适应学习的模式。习题推荐系统包括学习知识掌握情况建模和基于知识掌握情况的习题推荐两个部分。知识掌握建模往往利用学习者的学习交互记录例如做题记录、测验记录等等来对于学生的内在学习特征进行刻画,实现对于学习者知识掌握状态的动态追踪过程,以此构建针对性的知识强化练习。习题推荐部分则通过分析学生的知识掌握情况,推荐对于学生的知识掌握较弱的部分提高最合适的习题,实现因材施教,提高学习效果和效率。

%目前,高中数学学科存在知识点联系紧密,习题库庞大而混乱,学生不知道如何针对性地提高知识掌握程度等问题。因而本文以高中数学学科为研究背景,目的是提出一种实时追踪学生数学知识掌握情况,并依照学生的知识状态进行习题推荐的系统。该系统分为三个部分,第一个部分对习题进行知识点标注,作为推荐的依据,第二部分是核心部分,即基于学生的做题记录进行学生的知识掌握状态追踪,第三部分结合习题的知识点和学生的知识掌握情况进行习题推荐。该系统可以有效地实现自适应学习的目的。

The technical research and commercial application deployment of artificial intelligence have produced a trend of accelerated development in recent years. The deployment of various algorithms based on artificial intelligence and big data analysis has played an active role in accelerating the digitization of enterprises and organizations, improving the structure of the industrial chain, and increasing the efficiency of information utilization. As a traditional service industry, the education industry also has considerable room for intelligent improvement. In the traditional education model, students often take the class as the basic education collective. Therefore, in a class, the granularity of all teaching activities is also based on the class as the basic unit, which leads to the fact that the content of the students' learning does not completely match their needs, and what they have already mastered Excessive practice of knowledge points and lack of practice of unmastered knowledge points occur from time to time. This has caused students to be tired of learning, learning anxiety, etc., which has a greater impact on learning efficiency and learning effects. On the other hand, for teachers, generalized detailed monitoring and evaluation of individual knowledge status requires a huge workload, so teachers often only pay attention to a part of the students, leading to the neglect of the learning situation of most students. This has a greater impact on students' learning enthusiasm and learning conditions.

In recent years, China has issued a series of policies to promote the application of artificial intelligence in education. Using artificial intelligence technology to monitor and track student learning can greatly improve the quality and efficiency of education, thereby contributing to the realization of smart education. In smart education, adaptive learning is a business model that has undergone a lot of practice and verification. Quite a number of online education platforms have deployed adaptive learning system services. It uses data analysis, deep learning and other technologies to analyze students' learning behavior data, automatically evaluate and track students' knowledge status, and combine students' potential, advantages and other personalized information to provide students with learning path planning and learning resource recommendation services . It can help students improve the effectiveness and efficiency of learning, and improve the teaching quality of teachers while reducing teachers' teaching pressure. Among them, the exercise recommendation system is an adaptive learning model. The exercise recommendation system includes two parts: modeling of learning knowledge mastery and exercise recommendation based on knowledge mastery. Knowledge mastering modeling often uses learners' interactive learning records, such as question records, test records, etc., to characterize the students' internal learning characteristics, and to realize the dynamic tracking process of the learner's knowledge mastery status, so as to build targeted knowledge Intensive exercises. The exercise recommendation part analyzes the students' knowledge mastery, recommends the most suitable exercises for the parts with weaker knowledge mastery of the students, so as to realize teaching students in accordance with their aptitude and improve the learning effect and efficiency.

At present, high school mathematics subjects are closely connected with knowledge points, the question bank is huge and chaotic, and students do not know how to improve their knowledge mastery in a targeted manner. Therefore, this article takes the high school mathematics subject as the research background, and the purpose is to propose a real-time tracking student's mathematics knowledge mastery, and according to the student's state of knowledge for exercise recommendation system. The system is divided into three parts. The first part is to mark the knowledge points of the exercises as a basis for recommendation. The second part is the core part, which is to track the students' knowledge mastery status based on the students' record of the exercises. The third part combines The knowledge points of the exercises and the students' knowledge mastery are recommended for exercises. The system can effectively achieve the purpose of adaptive learning.

%********************************** %Second Section  *************************************
\section{Research Status}
%本文的研究课题为高中数学习题推荐系统,第一个部分为基于图神经网络和自然语言处理的习题知识点标注,第二个部分为基于图注意力网络与Transformer的知识追踪模型,第三个部分为基于协同过滤和神经网络的习题推荐模块, 的知识跟踪与推荐系统等。因而本文涉及的技术包括高中数学图神经网络,自然语言处理,多标签分类,推荐系统等研究课题, 接下来,本节将回顾这些技术的研究现状。

The research topic of this article is a high school math learning question recommendation system. The first part is the labeling of exercise knowledge points based on graph neural network and natural language processing. The second part is the knowledge tracking model based on graph attention network and Transformer. The third part is the knowledge tracking model based on graph attention network and Transformer. Part of it is the exercise recommendation module based on collaborative filtering and neural network, the knowledge tracking and recommendation system, etc. Therefore, the technologies involved in this article include high school mathematical graph neural networks, natural language processing, multi-label classification, recommendation systems and other research topics. Next, this section will review the current research status of these technologies.

\subsection{Property of high school Math}
%学科和知识彼此密切相关,因此学科知识表示包含在特定研究领域中的特定知识。在本研究中,学科仅针对教育领域的特定学科,例如数学,语言,化学等。第一步是学习如何充分利用现有知识。知识是从实践中获得的,因此在学习之后,也可以将其应用于社会实践。科学知识是说明性的,因为它可以用一系列符号,文字和图表表示;它也是程序性的,因为它可以在具体学习过程中根据特定的逻辑顺序进行安排和学习。

%数学是一门专门研究数量与空间形式之间关系的科学,其符号系统更加完整,公式结构清晰独特,文字和图像等语言表达方式也更加生动直观。

%学习者最需要学习的知识来自其前辈在实践活动中的经验总结。学习过程是对总结的知识进行认知学习,对知识结构进行不断的消化,调整和重组,以建立更加完善和合适的知识结构的过程,也是与创新思维相结合的过程。因此,良好的认知结构可以促进知识结构的形成,而良好的知识结构可以丰富认知结构的组织形式。由于学科知识结构由知识构成和知识依存性两部分组成,因此我们将从知识结构和构成这两个方面来分析学科知识结构。


Disciplines and knowledge are closely related to each other, so that disciplinary knowledge denotes the specific knowledge contained in a particular field of study. Disciplines are referred to in this study only for specific subjects in the field of education, such as mathematics, language, chemistry and so on. The first step is to learn how to make the best use of the knowledge that is available. The knowledge is obtained from practice, so after learning it, it can also be applied to social practice. Scientific knowledge is declarative because it can be expressed in a series of symbols, words and diagrams; it is also procedural because it can be arranged and learned according to a specific logical order in the process of concrete learning.

Mathematics is a science specializing in the study of the relationship between quantities and spatial forms, its symbolic system is more complete, the formula structure is clear and unique, text and images and other expressions of language is also more vivid and intuitive.

The knowledge that learners need to learn mostly comes from the summaries of the experiences of their predecessors in practical activities. The learning process is a process of cognitive learning of the summarized knowledge and continuous digestion, adjustment and reorganization of the knowledge structure, so as to build a more perfect and suitable knowledge structure, as well as a process of integration with innovative thinking. Thus a good cognitive structure can promote the formation of knowledge structure, and a good knowledge structure can enrich the organization form of cognitive structure. Since the disciplinary knowledge structure consists of two parts: knowledge composition and knowledge dependency, we will analyze the disciplinary knowledge structure from these two aspects, knowledge structure and composition.

\subsection{Graph Neural Network}


\subsection{Knowledge tracing algorithms}
Knowledge Tracing is a technique that models students' knowledge acquisition based on their past answers to obtain a representation of their current knowledge state. The task is to automatically track the change of students' knowledge level over time based on their historical learning trajectory, in order to be able to accurately predict the students' performance in future learning and to provide appropriate learning tutoring. In this process, the knowledge space is used to describe the level of student knowledge acquisition. A knowledge space is a collection of concepts, and a student's mastery of a part of a collection of concepts constitutes the student's mastery of knowledge. Some educational researchers argue that students' mastery of a particular set of related knowledge points will affect their performance on the exercise, i.e., the set of knowledge that students have mastered is closely related to their external performance on the exercise.


\subsection{Recommendation System}


%********************************** % Third Section  *************************************
\section{Research Objectives and Content}  %Section - 1.3
%本研究的目的是建立基于知识跟踪和因子分解机算法的高中数学学习资源推荐系统。我们使用知识跟踪对学生的知识状态进行建模,然后输出图形化的知识状态向量,将其用作下一级输入,同时考虑学生的个性化差异和知识遗忘过程,并将分解机算法应用于资源推荐系统。对于知识跟踪,我们建立了一个基于图神经网络的知识跟踪模型,该模型可以很好地表征数学科目中知识点的内在联系(考虑到知识点是一个类图结构),并输出一个图知识向量矩阵,还可以有效地刻画问题与知识点之间的联系。然后将知识跟踪模型的输出通过分解机算法,以获取学习资源的推荐度,并输出针对不同学习资源的推荐权重向量。

\section{Thesis Organization and Structure}
%本文的第一章是导论。 介绍了研究的研究背景,当前与行业相关的研究进展和研究重点。然后得出本文的三个核心点:习题知识点标注,知识跟踪和习题推荐。本文的第2章着重于习题知识点标注,在习题资源推荐系统中,需要对解析出习题的知识点,然后根据学生当前的知识掌握情况,针对性地推荐学生掌握不足的知识点相关习题。本章节的算法模型分为习题文本信息抽取和标签标注两个部分,在实验部分通过与若干传统模型进行对比来验证模型的有效性。本文的第3章提出了一个基于图神经网络的知识追踪模型,该模型分为习题-知识点关系嵌入学习、知识状态编码和答题预测解码三个部分。文中先对设计思路和相关技术进行理论介绍,然后在实验部分通过在公开数据集上与基准模型进行对比,验证模型的有效性并评估模型性能。本文的第4章提出了一种基于资源召回和资源排序两个阶段的推荐系统模型,该模型的召回阶段,通过协同过滤算法过滤出学习状况相似学生的相关习题,然后基于习题知识点标注模块所标注的习题知识点向量和知识追踪模块的知识状态隐含表征向量,设计一个神经网络模型来进行习题优先级排序。本文的第5章提出了结论和对于模型各个部分的未来的改进方向。


Chapter 1 of this paper is the introduction. Introduces the research background of the research, the research progress and research focus related to the algorithm used in this article. By analyzing the requirements of the exercise recommendation system, three core points of this article are drawn: learning resource representation, knowledge tracing and resource recommendation.

Chapter 2 of this article focuses on the labeling of exercise knowledge points. In the exercise resource recommendation system, the knowledge points of the exercises need to be parsed, and then based on the students' current knowledge mastery, it is recommended that the students have insufficient knowledge points and related exercises. The algorithm model in this chapter is divided into two parts: text information extraction of exercises and labeling. In the experimental part, the effectiveness of the model is verified by comparing with several traditional models.

Chapter 3 of this article proposes a knowledge tracing model based on graph neural network. The model is divided into three parts: exercise-knowledge point relationship embedding learning, knowledge state encoding and answer prediction decoding. The article first introduces the design ideas and related technologies theoretically, and then compares the benchmark model with the public data set in the experimental part to verify the effectiveness of the model and evaluate the performance of the model.

Chapter 4 of this article proposes a recommendation system model based on the two stages of resource recall and resource ranking. In the recall stage of the model, the relevant exercises of students with similar learning conditions are filtered out through collaborative filtering algorithms, and then the module is labeled based on exercise knowledge points The labeled exercise knowledge point vector and the knowledge state implicit representation vector of the knowledge tracing module are designed to design a neural network model to prioritize exercises.

Chapter 5 of this article puts forward conclusions and directions for future improvements in each part of the model.

% \nomenclature[z-DEM]{DEM}{Discrete Element Method}
% \nomenclature[z-FEM]{FEM}{Finite Element Method}
% \nomenclature[z-PFEM]{PFEM}{Particle Finite Element Method}
% \nomenclature[z-FVM]{FVM}{Finite Volume Method}
% \nomenclature[z-BEM]{BEM}{Boundary Element Method}
% \nomenclature[z-MPM]{MPM}{Material Point Method}
% \nomenclature[z-LBM]{LBM}{Lattice Boltzmann Method}
% \nomenclature[z-MRT]{MRT}{Multi-Relaxation
% 	Time}
% \nomenclature[z-RVE]{RVE}{Representative Elemental Volume}
% \nomenclature[z-GPU]{GPU}{Graphics Processing Unit}
% \nomenclature[z-SH]{SH}{Savage Hutter}
% \nomenclature[z-CFD]{CFD}{Computational Fluid Dynamics}
% \nomenclature[z-LES]{LES}{Large Eddy Simulation}
% \nomenclature[z-FLOP]{FLOP}{Floating Point Operations}
% \nomenclature[z-ALU]{ALU}{Arithmetic Logic Unit}
% \nomenclature[z-FPU]{FPU}{Floating Point Unit}
% \nomenclature[z-SM]{SM}{Streaming Multiprocessors}
% \nomenclature[z-PCI]{PCI}{Peripheral Component Interconnect}
% \nomenclature[z-CK]{CK}{Carman - Kozeny}
% \nomenclature[z-CD]{CD}{Contact Dynamics}
% \nomenclature[z-DNS]{DNS}{Direct Numerical Simulation}
% \nomenclature[z-EFG]{EFG}{Element-Free Galerkin}
% \nomenclature[z-PIC]{PIC}{Particle-in-cell}
% \nomenclature[z-USF]{USF}{Update Stress First}
% \nomenclature[z-USL]{USL}{Update Stress Last}
% \nomenclature[s-crit]{crit}{Critical state}
% \nomenclature[z-DKT]{DKT}{Draft Kiss Tumble}
% \nomenclature[z-PPC]{PPC}{Particles per cell}
