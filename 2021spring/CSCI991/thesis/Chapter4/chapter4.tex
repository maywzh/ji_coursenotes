\chapter{Exercise Recommendation System Based on Knowledge State }

% **************************** Define Graphics Path **************************
\ifpdf
    \graphicspath{{Chapter4/Figs/Raster/}{Chapter4/Figs/PDF/}{Chapter4/Figs/}}
\else
    \graphicspath{{Chapter4/Figs/Vector/}{Chapter4/Figs/}}
\fi

\section{Motivation}

%自适应学习是指学习者根据学习内容的不同,选择不同的学习方式,在学习中 不断发现,最终找到符合自己的“个性化”学习方式。自适应学习不是简单的将线下资源搬到线上,也不是简单的“互联网+教育”的结合,而是注重以学生个体为 中心,对具有不同认知水平、不同认知风格的人提供不同的个性化学习方案。

%自适应学习概念一直是教育界研究的热点,也是很多教育学家研究的理论之 一。自从 Brusilovsky 提出自适应学习系统的通用模型开始,自适应学习概念才第一 次应用到工程实践当中,但是一直缺乏标准的参考模型和架构体系,直到第一个被 开发出来的自适应超媒体应用模型,到后来有 LAOS、XML 自适应媒体模型和WebXML 参考模型,自此,完成了自适应学习从教育界理论研究向计算机领域的 应用开发转变,众多自适应学习模型都将关注点聚焦在领域模型、学生模型和自适 应引擎的构建上。

%领域模型主要是构建学生的学习资源,此处的学习资源不是单纯的学生教材、 辅导资料,而是将该领域知识中的各个实体进行有结构化的组织,以本体或语义网 的形式来构建实体之间的关联,形成完备的知识库,其中实体主要包括细粒度的知 识点概念及其属性、例题、习题、资料甚至图形图像和音频等。

%学生模型主要是用来构建具有学生个体特征的模型,可以描述为学生个体对 领域模型中每个试题的掌握情况,学生模型会随着领域模型中实体关系的变化而 转变,学生个体特征还包括了学生个体的兴趣偏好、测试成绩、认知能力和知识水 平等。除此之外,根据我国教育技术规范关于学习者规范模型 CELTS-11中也对学 生模型进行了描述,增加了学生的年龄、背景、地区等额外信息。

%自适应引擎又被称为教学选择策略,通过对领域模型到学生模型的自适应解 释规则,实现能够根据学生的个性特征,由系统通过相关规则或是算法选择出满足 学生要求的领域模型供学生学习。目前较为流行的是运用大数据技术,通过统计学 生学习的路径或习惯,采用协同过滤推荐、内容推荐、混合推荐等流行的推荐算法, 推荐学生个性化的学习方案。

%本文在深入理解自适应参考模型后,进一步优化了模型中领域模型、学生模型、 教学策略,并将优化后的模型运用到构建智能题库系统上。将特征题库按照遗忘度、 抽象度、间接度、综合度、计算复杂度等维度进行信息结构化分类,学生个体通过 训练实时生成反映学生知识掌握能力的个体特征库。运用智能选题算法以个体特 征库为基础,动态从题库中智能选取切合个体的训练集。

High school mathematics intelligent question bank is built by learning from the model of adaptive learning. Adaptive learning is the theoretical abstraction of adaptive learning system model. Many adaptive learning system models focus on the construction of domain model, student model and adaptive engine. This paper proposes an intelligent recommendation algorithm based on individual characteristics, which integrates domain model, student model and adaptive engine The self-adaptive engine abstracts the high school mathematics question bank into the characteristic question bank according to the domain model, and abstracts the student model into the individual characteristic bank. The self-adaptive engine is based on the individual characteristic bank through the intelligent algorithm, dynamically selects the training questions suitable for the individual from the characteristic question bank, and realizes the personalized recommendation of the question bank. 

Traditional paper teaching materials are usually printed in large quantities. Due to the limitation of paper space and in order to adapt to the majority of students, one or two representative test questions have to be selected according to each chapter. In order to meet the comprehensiveness of knowledge points and the public adaptability, such teaching aids lead to the small scale of test questions, the incomplete coverage of questions and the failure to meet the requirements of students' targeted training. E-learning has brought a reform in the field of teaching with its flexible learning style, avoiding large class teaching, and students can choose their own learning resources through autonomous navigation. Online question bank with eleaning As the times require, the early online question bank simply moves the offline resources to the Internet. Although students can choose their own topics and find the test questions they want for intensive training, it is lack of interactivity. Students do not get timely and effective feedback after finishing the questions, and can not accurately locate the strengths and weaknesses of students, which is very limited in improving students' performance. In view of these shortcomings, it is the current research direction of intelligent question bank to build a student-centered system, in which students can feed back their mastery of knowledge points in real time by doing questions, and build a complete user portrait of students. The question bank can intelligently diagnose, screen and push targeted questions according to the submission of students' questions.

Adaptive learning means that learners choose different learning styles depending on the content they are learning, and that they continue to discover their own "personalized" learning style as they learn. Adaptive learning is not a simple transfer of offline resources to online, nor is it a simple combination of "Internet + education". Rather, it focuses on the individual student and provides different personalized learning programs for people with different cognitive levels and styles.

The concept of adaptive learning has been a hot topic of research in the education field and has been one of the theories studied by many educators. Since Brusilovsky proposed a general model of adaptive learning system, the concept of adaptive learning was first applied to engineering practice, but there has been a lack of standard reference models and architectural systems until the first adaptive hypermedia application model was developed, and later there were LAOS, XML adaptive media model and WebXML reference model.
Since then, adaptive learning has been transformed from theoretical research in education to application development in computing, and many adaptive learning models have focused on the construction of domain models, student models, and adaptive engines.

The domain model focuses on building student learning resources, which are not just student textbooks or tutorials, but also structured entities in the domain knowledge, and ontologies or semantic webs are used to build connections between entities to form a complete knowledge base, including fine-grained knowledge concepts and their attributes, examples, exercises, materials, and even graphical images and audio.

The student model is mainly used to build a model with individual student characteristics, which can be described as individual students' mastery of each test topic in the domain model, and the student model will change with the changes of entity relationships in the domain model. In addition, the student model is also described in CELTS-11 according to our educational technology specification on the learner specification model, which adds additional information such as the student's age, background, and region.

The adaptive engine, also known as the instructional selection strategy, uses adaptive interpretation rules from the domain model to the student model to enable the system to select the domain model that meets the student's requirements based on the student's personality characteristics, either through rules or algorithms. Currently, it is popular to use big data technology to recommend students' personalized learning solutions by using popular recommendation algorithms such as collaborative filtering recommendation, content recommendation, and hybrid recommendation through statistics of students' learning paths or habits.

In this paper, we further optimize the domain model, student model, and teaching strategy in the adaptive reference model, and apply the optimized model to build an intelligent question bank system. The feature database is structured according to the dimensions of forgetfulness, abstractness, indirectness, comprehensiveness, and computational complexity, and individual students are trained to generate a database of individual features reflecting their knowledge acquisition ability in real time. The intelligent question selection algorithm is used to dynamically select the appropriate training set from the question database based on the individual feature database.

\section{Proposed Model}
\subsection{Algorithm Overview}
