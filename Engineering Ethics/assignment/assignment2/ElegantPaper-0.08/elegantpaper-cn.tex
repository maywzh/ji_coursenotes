%!TEX program = xelatex
% 完整编译: xelatex -> bibtex -> xelatex -> xelatex
\documentclass[lang=cn,11pt,a4paper,twocolumn]{elegantpaper}

\title{RoBERTa:鲁棒优化的BERT预训练方法}
\author{Yinhan Liu$^{\S}$ \and Myle Ott$^{\S}$ \and Naman Goyal$^{\S}$ \and Jingfei Du$^{*\S}$ \and Mandar Joshi{$^\dag$} \and Danqi Chen$^\S$ \and Omer Levy$^\S$ \and Mike Lewis$^\S$ \and Luke Zettlemoyer$^{\dag\S}$ \and Veselin Stoyanov$^\S$}
\institute{$^\dag$Paul G. Allen School of Computer Science \& Engineering,\\ University of Washington, Seattle, WA\\ $^\S$Facebook AI}

\version{}
\date{}

\begin{document}
\maketitle
\begin{abstract}
  语言模型的预训练带来了显着的性能提升,但是不同方法之间的仔细比较却具有挑战性。 训练的计算量很大,这通常是在不同大小的私有数据集上进行,而且正如我们将要表明的,超参数的选择对最终结果有重大影响。 我们提出了BERT预训练的再研究(Devlin等,2019),该研究仔细衡量了许多关键超参数和训练数据量的影响。 我们发现BERT的训练不足,并且可以匹配或超过其发布的每个模型的性能。 我们最好的模型在GLUE,RACE和SQuAD上获得了最优的结果。 这些结果突出了以前被忽视的设计的重要性,并引起了人们对最近报道改进的来源的质疑。 我们发布我们的模型和代码。
% \keywords{Elegant\LaTeX{},工作论文,模板}
\end{abstract}

\section{介绍}
自训练方法,例如ELMo(Peters等,2018),GPT(Radford等,2018),BERT(Devlin等,2019),XLM(Lample和Conneau,2019)和XLNet(Yang等,2019年)取得了显着的性能提升,但要确定方法的哪些方面贡献最大可能是具有挑战性的。训练在计算上是代价较大的,限制了可以完成的调整量,并且经常使用大小不同的私有训练数据来进行,从而限制了我们衡量建模优化效果的能力。

我们提出了BERT预训练的再研究(Devlin等,2019),其中包括对超参数调整和训练集大小的影响的仔细评估。 我们发现BERT的训练不足,并提出了一种改进的方法来训练BERT模型(我们称为RoBERTa),该模型可以匹敌或超过所有后BERT方法的性能。我们的修改很简单,其中包括:(1)在更多数据上以更大的批次训练模型更长的时间;(2)删除下一句预测目标;(3)对更长序列的训练;(4)动态改变应用于训练数据的遮罩模式。 我们还收集了一个与其他私有数据集大小相当的大型新数据集(CC-NEWS),以更好地控制训练集大小的效果。

当控制训练数据时,我们改进的训练程序会改进GLUE和SQuAD上已发布的BERT结果。经过更多数据训练后,我们的模型在GLUE公开排行榜上得分为88.5,于Yang等人得分为88.4的报告(2019)匹敌。  我们的模型针对9个GLUE任务中的4个建立了新的最新技术:MNLI,QNLI,RTE和STS-B。 我们还将SQuAD和RACE上的最新结果进行匹配。总体而言,我们重新确定了BERT的掩蔽语言模型训练目标可与最近提出的其他训练目标(例如受干扰的自回归语言建模(Yang等人,2019)))竞争。

总而言之,本文的贡献是:(1)提出了一组重要的BERT设计和训练策略,并介绍了可以改善下游任务性能的替代方法; (2)我们使用新的数据集CC-NEWS,并确认使用更多数据进行预训练可进一步提高下游任务的性能; (3)我们的训练改进表明,在正确的设计选择下,遮罩语言模型预训练与所有其他最近发布的方法相比具有竞争力。 我们发布了我们的模型,在PyTorch中实现了预训练和微调代码(Paszke等,2017)。

\section{背景}
在本节中,我们简要概述了BERT(Devlin等人,2019)的预训练方法以及我们将在下一节中通过实验测试的一些训练选择。
\subsection{配置}
BERT将两个段(令牌序列)$x_1,...,x_N$和$y_1,...,y_M$的串联作为输入。段通常包含一个以上的自然句子。这两个段作为单个输入序列呈现给BERT,并用特殊的记号分隔它们:$[CLS],x_1,...,x_N,[SEP],y_1,...,y_M,[EOS]$。$M$和$N$被约束为$M+N<T$,其中$T$是控制训练期间最大序列长度的参数。

该模型首先在庞大的无标签文本语料库上进行了预训练,然后使用最终任务标记的数据进行了微调。

\subsection{架构}
BERT使用了如今无处不在的变换器架构(Vaswani等,2017),我们将不对其进行详细介绍。 我们使用具有$L$层的变换器架构。 每个块都使用$A$自感知头和隐藏维$H$。

\subsection{训练目标}
在预训练期间,BERT使用两个目标:遮罩语言建模和下一句预测。
\subsubsection{遮罩语言模型(MLM)}
选择输入序列中令牌的随机样本,并将其替换为特殊令牌$[MASK]$。 MLM的目标是预测掩盖令牌时的交叉熵损失。 BERT统一选择15\%的输入令牌以进行替换。 在所选令牌中,有80\%被替换为$[MASK]$,10\%则保持不变,还有10\%被随机选择的词典令牌代替。

在最初的实现中,随机遮罩和替换从头开始执行一次,并在训练期间保存,尽管在实践中,数据是重复的,所以每个训练语句的遮罩并不总是相同的(参见第4.1节)。
\subsubsection{下句预测(NSP)}
NSP是二进制分类损失,用于预测原始文本中两个段是否紧随其后。正样例是通过从文本语料库中选取连续的句子来创建的。负样例是通过将不同文档中的句段配对而创建的。 正样本和负样本均以相同的概率采样。

\subsection{优化}

\subsection{数据}

\section{实验配置}

\subsection{实现}

\nocite{*}
\bibliography{wpref}

\end{document}
