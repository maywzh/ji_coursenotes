\documentclass[lang=cn,11pt,a4paper]{elegantpaper}

\title{深圳--中国特色社会主义先行示范区之意义与方向}
\author{梅王智汇 2019124044}
\institute{伍伦贡联合研究院}

% \version{0.08}
\date{}

\begin{document}

\maketitle

% \begin{abstract}

% \keywords{Elegant\LaTeX{},工作论文,模板}
% \end{abstract}


\section{改革开放试验田-深圳}
南中国有一座城市,高度浓缩一个时代精华,如同施展了法术,40年砥砺奋进,从一个默默无闻的边陲小镇到拥有2000万人的现代化国际都市,GDP猛增1.1万倍,常住人口增加近40倍。是什么成就了深圳?是改革开放让深圳焕发出前所未有的生命力,有力地证明了中国特色社会主义一定会越走越宽广,让世界看到了石破天惊的中国速度,感慨这场前无古人的中国奇迹。

2019年8曰25日,中共中央、国务院发布了《中共中央 国务院关于支持深圳建设中国特色社会主义先行示范区的意见》。强化了深圳作为中国特色社会主义建设先锋的地位。这是以习近平同志为核心的党中央立足新时代中国特色社会主义建设大局作出的一项重大决策,既体现了党中央对深圳改革开放取得成就的充分肯定,又饱含着对深圳在新时代继续深化改革开放的殷切期望,为深圳未来发展提供了强大动力和根本遵循。

\section{意义}
\subsection{深度推进改革开放}
有利于在更高起点、更高层次、更高目标上推进改革开放,形成全面深化改革、全面扩大开放新格局。改革开放是中国命运的一次重大转折,是党在关键时刻的一次重大自我觉醒。1980年,党中央决定兴办深圳经济特区,就是要发挥其对全国改革开放和社会主义现代化建设的重要窗口和示范作用。近四十年来,深圳取得了巨大的经济成就,实现了天翻地覆的变化,为探索中国特色社会主义道路作出了重大贡献。当前,我国的改革开放又到了一个新的历史关头,推进改革开放的复杂程度和艰巨程度不亚于四十年前。党中央支持深圳建设中国特色社会主义先行示范区,就是希望深圳一如既往地当好改革开放尖兵,为新时代改革开放再出发探出新路。

\subsection{粤港澳大湾区战略}
有利于更好实施粤港澳大湾区战略,丰富“一国两制”事业发展新实践。在粤港澳大湾区建设全面推进的关键阶段,党中央支持深圳建设中国特色社会主义先行示范区,就是要求深圳全面深化规则机制对接,要求在粤港澳大湾区中利用制度互补优势;要求深圳增强核心引擎功能,更好地辐射带动其他湾区城市加快发展。

\subsection{探索全面建设社会主义现代化强国新路径}
有利于率先探索全面建设社会主义现代化强国新路径,为实现中华民族伟大复兴的中国梦提供有力支撑。新时代中国特色社会主义的鲜明主题,就是为全面建成社会主义现代化强国和实现中华民族伟大复兴中国梦而奋斗。这要求锐意进取、大胆探索,不断有所发现、有所创造、有所前进。深圳是一座国际化创新型城市,全国现代化进程中遇到的问题,很大可能在深圳最先出现。党中央支持深圳建设中国特色社会主义先行示范区,就是要求深圳瞄准创建社会主义现代化强国城市范例的目标,在建设社会主义现代化国家的新征程中,用实践来作出新贡献。

\section{方向}
\subsection{把握根本目标}
准确把握创建社会主义现代化强国城市范例这一根本目标。中国是世界上唯一一个不经过资本主义而实现国家现代化的大国,深圳是中国特色社会主义事业发展的重要起源地和践行地,深圳的实践为这一事业的形成和探索提供了丰富素材和鲜活经验,创造了可复制可推广的成功样板和典型范例。迈进新时代,深圳建设先行示范区,道路只有一条,就是坚持和发展中国特色社会主义;目标只有一个,就是创建社会主义现代化强国的城市范例。

\subsection{贯彻根本要求}
贯彻落实先行示范这一根本要求。“先行”要求的是以敢为人先勇于试错容错,以“一子突破”求得“全盘皆活”,围绕亟须突破的重点和难点敢闯敢试,为全国改革开放再出发探索新路子。“示范”体现的是以“一马当先”带动“万马奔腾”,以一域服务全局,形成更多可复制、可推广的经验和制度,发挥对周边地区乃至全国的引领带动作用。“先行示范”强调的是全方位、全过程的“先行示范”,不只是满足争当“单项冠军”,而是要按照“五位一体”的总体布局,努力成为高质量发展高地、法治城市示范、城市文明典范、民生幸福标杆和可持续发展先锋。

\subsection{深化根本动力}
着力增强深化改革开放这一根本动力。改革开放是决定当代中国命运的关键一招,也是深圳由一个小渔村发展成国际化大都市的主要驱动力量。今天,建设中国特色社会主义先行示范区,深圳仍然要依靠改革开放,为全国深化市场化改革和扩大高水平开放探索新路。要围绕要素市场化配置、营商环境优化、城市空间统筹利用等重点领域,深入推进市场化改革试点。要围绕自由贸易试验区、对港对澳合作等重大开放平台建设,探索实施更大范围、更大力度的高水平开放举措。

\subsection{加强根本保障}
全面加强党的领导和党的建设这一根本保障。中国共产党的领导是中国特色社会主义最本质的特征,也是中国特色社会主义制度的最大优势。深圳建设的先行示范区,是中国共产党领导的先行示范区,是中国特色社会主义的先行示范区。在先行示范区建设的过程中,要坚持和发挥党的领导核心作用,把政治建设放在首位,确保先行示范区建设的方向不偏;要坚持和加强党对经济工作的领导,使市场在资源配置中起决定性作用,更好发挥政府作用,增强先行示范区建设的发展活力;要坚持和加强全面从严治党,贯彻落实新时代党的组织路线,激励特区干部新时代新担当新作为,强化先行示范区建设的干部人才队伍保障。
\end{document}