@Inbook{Furnkranz2010,
  author="F{\"u}rnkranz, Johannes",
  editor="Sammut, Claude
  and Webb, Geoffrey I.",
  title="Machine Learning and Game Playing",
  bookTitle="Encyclopedia of Machine Learning",
  year="2010",
  publisher="Springer US",
  address="Boston, MA",
  pages="633--637",
  isbn="978-0-387-30164-8",
  doi="10.1007/978-0-387-30164-8_504",
  url="https://doi.org/10.1007/978-0-387-30164-8_504",
  annotate={Machine learning method has been applied into game-playing for a long period. Video game industry has found game AI as an important role for entertainment and enjoyness of the game product, while computer game is successful application of AI. Games offer oppotunities for AI development such as simulation, data analysis and education. Topics about machine learning applying into game-playing include learning player and game, model learning, performance and so on.}
}


@inproceedings{Yannakakis:2012:GAR:2212908.2212954,
  author = {Yannakakis, Geogios N.},
  title = {Game AI Revisited},
  booktitle = {Proceedings of the 9th Conference on Computing Frontiers},
  series = {CF '12},
  year = {2012},
  isbn = {978-1-4503-1215-8},
  location = {Cagliari, Italy},
  pages = {285--292},
  numpages = {8},
  url = {http://doi.acm.org/10.1145/2212908.2212954},
  doi = {10.1145/2212908.2212954},
  acmid = {2212954},
  publisher = {ACM},
  address = {New York, NY, USA},
  keywords = {game AI flagships, game artificial intelligence, game data mining, player experience modeling, procedural content generation},
  annotate={ The paper present four research fields on game artificial intelligence other than traditional non player character(NPC) control, consisting player experience modeling (PEM), procedural content generation (PCG), massive-scale game data mining and enhancing NPC AI. PEM construct computational models of experience of players, including subjective PEM, objective PEM, gameplay-based PEM and personalizing PEM. PCG research game content automatical generation. Data mining over massive-scale games can perform data analysis to game to evaluate game and improve game quality. AI enhancement improve the enjoyness and immersion of gameplay.}
} 


@ARTICLE{7366548,
  author={M. {Frutos-Pascual} and B. G. {Zapirain}},
  journal={IEEE Transactions on Computational Intelligence and AI in Games},
  title={Review of the Use of AI Techniques in Serious Games: Decision Making and Machine Learning},
  year={2017},
  volume={9},
  number={2},
  pages={133-152},
  abstract={The video game market has become an established and ever-growing global industry. The health of the video and computer games industry, together with the variety of genres and technologies available, means that video game concepts and programmes are being applied in numerous different disciplines. One of these is the field known as serious games. The main goal of this paper is to collect all the relevant articles published during the last decade and create a trend analysis about the use of certain artificial intelligence algorithms related to decision making and learning in the field of serious games. A categorization framework was designed and outlined to classify the 129 papers that met the inclusion criteria. The authors made use of this categorization framework for drawing some conclusions regarding the actual use of intelligent serious games. The authors consider that over recent years enough knowledge has been gathered to create new intelligent serious games to consider not only the final aim but also the technologies and techniques used to provide players with a nearly real experience. However, researchers may need to improve their testing methodology for developed serious games, so as to ensure they meet their final purposes.},
  keywords={decision making;learning (artificial intelligence);serious games (computing);AI techniques;decision making;machine learning;video game market;computer games industry;artificial intelligence algorithms;categorization framework;intelligent serious games;Games;Artificial intelligence;Decision making;Market research;Algorithm design and analysis;Decision trees;Industries;Artificial intelligence;games;intelligent systems;machine learning},
  doi={10.1109/TCIAIG.2015.2512592},
  ISSN={},
  month={June},
  annotate={Serious game is a kind of video game which are designed for certain purposes other than enjoyment. The paper collected several relevant articles of past decade and analysed the trend of artificial intelligence method related to decision making and machine learning about serious game. The author also discussed algorithms used as well as factors of market, purpose, platform and so on. By comparing different AI techniques in several serious games, the author draw the conlusion that AI in serious game provide players with a nearly real experience and still need better method for testing.}
}

@misc{chen2019learningbased,
    title={Learning-Based Video Game Development in MLP@UoM: An Overview},
    author={Ke Chen},
    year={2019},
    eprint={1908.10127},
    archivePrefix={arXiv},
    primaryClass={cs.AI},
    annotate={The paper focus on applying machine learning techniques to video game development, comparing which with traditional methodologies in in Machine Learning and Perception Lab at the University of Manchester(MLP@UoM). The author overviewd learning-based procedural content generation, learning-based serious education games. fast skill capture via learning and object-based learnable agents respectively. Finally, the paper discussed these techniques and draw the conclution that human-like learnable agents with machine learning methods would be important research area in the future.}
}

@InProceedings{lorch2007enhancing,
  author = {Lorch, Jay and Uyeda, Frank and Wood, Randall C. and Douceur, John (JD)},
  title = {Enhancing game-server AI with distributed client computation},
  booktitle = {Proceedings of the 17th International Workshop on Network and Operating Systems Support for Digital Audio and Video (NOSSDAV)},
  year = {2007},
  month = {June},
  abstract = {In the context of online role-playing games, we evaluate offloading AI computation from game servers to game clients. In this way, the aggregate resources of thousands of participating client machines can enhance game realism in a way that would be prohibitively expensive on a central server. Because offloading can add significant latency to a computation normally executing within a game server’s main loop, we introduce the mechanism of AI partitioning: splitting an AI into a high-frequency but computationally simple component on the server, and a lowfrequency but computationally intensive component offloaded to a client. By designing the client-side component to be stateless and deterministic, this approach also facilitates rapid handoff, preemptive migration, and replication, which can address the problems of client failure and exploitation. To explore this approach, we develop an improved AI for tactical navigation, a challenging task to offload because it is highly sensitive to latency. Our improvement is based on calculating influence fields, partitioned into server-side and client-side components by means of a Taylor series approximation. Experiments on a Quake-based prototype demonstrate that this approach can substantially improve the AI’s abilities, even with server-clientserver latencies up to one second.},
  publisher = {Association for Computing Machinery, Inc.},
  url = {https://www.microsoft.com/en-us/research/publication/enhancing-game-server-ai-with-distributed-client-computation/},
  pages = {31-36},
  edition = {Proceedings of the 17th International Workshop on Network and Operating Systems Support for Digital Audio and Video (NOSSDAV)},
  annotate={ In online role-playing games, offloading AI computation from game servers may cause heavy latency. The paper present AI computation with distributed client. The mechanism split server-side AI compuatation into server-side simple high-frequency part and client-side low-frequency intensive part. The author analysed the problem of latency and evaluate the method with example of tactical navigation. It showed practical result in offloading compuatation to client-side. }
}

