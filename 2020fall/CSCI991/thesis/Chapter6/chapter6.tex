\chapter{Conclusion and Future Work}

%在本文中,建立了一个基于图神经网络结合知识图谱以及因式分解算法的试题推荐系统,它总体上完成了设计目标,并在推荐系统的三个阶段-数据源生成、知识追踪和知识推荐等方面都具有独创性,本文采用当前较为热门的图神经网络模型,并对现有的模型进行了一些改进,来解决试题的知识分析和学生的知识状态追踪任务,并在实验中达到了理想的结果,对现有模型有了一定的性能改善。

%其中,在数据源部分,通过爬虫、人工输入等方式输入到初始数据库中,然后通过基于文本挖掘和图神经网络迭代学习的试题自动知识点分析模型来完成试题知识点标签挖掘,它们再通过一些知识图谱的构建手段建立了高中数学知识图谱,解决了“零资源”的问题。在对学生的知识追踪方面,本文创新性地将图自注意力网络应用于该任务,通过对于模型的改进和对于多个因素的考虑,该任务相对现有的模型已经有了一些改进,并在性能上有所提升。最后,通过对试题的知识打标签和对学生进行知识追踪,利用因式分解机算法来完成学习资源推荐,它解决了冷启动的问题,完成了自适应学习的设计目标。

%在未来的工作中,可以结合其他的图神经网络模型或者加入一些记忆力机制来解决模型的序列性问题。

In this paper, a test question recommendation system based on graph neural network combined with knowledge mapping and factorization algorithm is established, which generally accomplishes the design goals and is original in the three stages of the recommendation system-data source generation, knowledge tracking and knowledge recommendation, etc. This paper adopts the current more popular graph neural network model and makes some improvements to the existing model to solve the tasks of knowledge analysis of test questions and knowledge state tracking of students, and achieves the desired results in the experiments with some performance improvements to the existing model.

Among them, in the data source part, they are input to the initial database by crawlers and manual input, and then the automatic knowledge point analysis model of test questions based on text mining and graph neural network iterative learning is used to complete the knowledge point label mining of test questions, and they then establish the knowledge map of high school mathematics by some means of knowledge map construction, which solves the "zero resource " problem. In terms of the knowledge tracking of students, this paper innovatively applies the graph self-attentive network to this task, and through the improvement of the model and the consideration of several factors, the task has been improved somewhat compared with the existing model and has improved in performance. Finally, by labeling the knowledge of the test questions and tracking the students' knowledge, the factorization machine algorithm is used to complete the learning resource recommendation, which solves the cold start problem and accomplishes the design goal of adaptive learning.

In future work, the sequential nature of the model can be solved by combining other graph neural network models or adding some memory mechanisms.